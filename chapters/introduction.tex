\chapter{Úvod}

Informační systémy slouží pro sběr, zpracování, prezentaci a~distribuci dat. S~informačními systémy komunikují lidé různých oborů a~zaměření, kteří mnohdy nejsou příliš počítačově gramotní. Cílem každého rozhraní je zajistit relativně jednoduchou komunikaci člověka s~počítačem (webovou aplikací). Tento druh komunikace je dán sazbou textu, souborem piktogramů a~dalšími prvky, které pomáhají v orientaci.

Cílem práce je návrh a realizace uživatelského rozhraní informačního systému pro správu bakalářských a~diplomových prací. Informační systém je určen společnosti Red Hat jako nástroj pro zadání, kontrolu a~schvalování závěrečných prací vypsaných na partnerských fakultách. Systém zároveň poskytuje rozhraní pro studenty a vedoucí z registrovaných fakult.

První část práce se zabývá tvorbou tzv. \uv{drátěného modelu} webu (anglicky označováno jako \textit{wireframing}), který podává představu o rozvržení jednotlivých komponent na stránce. Následující část zahrnuje tvorbu grafického návrhu, který reprezentuje grafickou podobu uživatelského rozhraní -- jedná se o~vizuální stránku webové aplikace. Finální část představuje integraci grafického návrhu do implementace informačního systému (tj. prezentaci informací v~definované grafické podobě webovými prohlížeči).
