\chapter{Úvod}

Informační systémy jsou nezbytnou součástí moderní společnosti. Prostřednictvím jejich nástrojů lidé denně zpracovávají, ukládají a distribuují ohro-mné množství informací. Mezi uživatele informačních systémů patří lidé nejrůznějších oborů i národností, jejichž cílem je vždy snadná a produktivní práce -- míra uplatnění těchto požadavků je přitom závislá na kvalitě provedení uživatelských rozhraní. Jejich tvorba si tak žádá zvýšenou péči a stává se poměrně složitou disciplínou.

Cílem práce je návrh a realizace uživatelského rozhraní informačního systému pro správu bakalářských a diplomových prací. Informační systém je určen společnosti Red~Hat jako nástroj pro zadání a kontrolu závěrečných prací vypsaných na partnerských fakultách. Systém zároveň poskytuje rozhraní pro studenty a vedoucí z registrovaných fakult.

První část práce se zabývá tvorbou \uv{drátěného modelu} webu (anglicky se tento proces nazývá \textit{wireframing}), který reprezentuje rozložení jednotlivých komponent v uživatelském rozhraní aplikace. Následující část zahrnuje tvorbu grafického návrhu, který reprezentuje grafickou podobu prezentační vrstvy systému -- jedná se o vizuální stránku webové aplikace. Finální část se zaměřuje na integraci grafického návrhu do implementace informačního systému.
