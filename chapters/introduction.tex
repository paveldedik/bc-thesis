\chapter{Úvod}

Informační systémy slouží pro sběr, zpracování, prezentaci a~distribuci dat. S~informačními systémy komunikují lidé různých oborů a~zaměření, kteří mnohdy nejsou příliš počítačově gramotní. Cílem každého grafického rozhraní je zajistit relativně jednoduchou komunikaci člověka s~počítačem (webovou aplikací). Tento druh komunikace je dán sazbou textu, souborem piktogramů a~dalšími prvky.

Bakalářská práce se zaměřuje na analýzu, návrh a~implementaci informačního systému pro správu bakalářských a~diplomových prací. Informační systém slouží především firmě Red Hat jako nástroj pro zadání, kontrolu a~schvalování závěrečných prací vypsaných na partnerských fakultách. Systém zároveň poskytuje rozhraní pro studenty a vedoucí.

V~první fázi je potřeba vytvořit tzv. \uv{drátěný model webu} (anglicky \textit{wireframe}), který reprezentuje rozvržení jednotlivých komponent na stránce. Drátěný model vychází z~požadavků zadavatele na systém a~prezentuje konkrétní funkce a~nástroje systému. Následující fáze zahrnuje tvorbu grafického návrhu, který reprezentuje grafickou podobu uživatelského rozhraní -- jedná se o~vizuální stránku webové aplikace. Poslední část tvoří kódování grafického návrhu tj. prezentaci informací v~definované grafické podobě webovými prohlížeči.

Vytvoření informačního systému je jednou z~nejsložitějších programátorských a~designérských disciplín. Ve většině případů pokračuje vývoj na systému i~po jeho uvedení do produkce. Některé systémy vyžadují velmi podrobnou analýzu, která není vždy jednoduše proveditelná především z~důvodu špatné komunikace uživatelů systému s~analytiky.
