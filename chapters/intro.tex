\chapter{Úvod}

Informační systémy slouží pro sběr, zpracování, prezentaci a~distribuci dat. S~informačními systémy komunikují lidé různých oborů a~zaměření, kteří mnohdy nejsou příliš počítačově gramotní. Z~toho důvodu je jednou z~důležitých vlastností informačních systémů poskytovat intuitivní uživatelské rozhraní. Cílem každého grafického rozhraní je zajistit relativně jednoduchou komunikaci člověka s~počítačem (webovou aplikací). Tento druh komunikace je dán sazbou textu, souborem piktogramů a~dalšími prvky, které napomáhají lidem v~orientaci.

Bakalářská práce se zaměřuje na analýzu, návrh a~implementaci informačního systému pro správu bakalářských a~diplomových prací. Informační systém slouží především firmě Red Hat jako nástroj pro zadání, kontrolu a~schvalování závěrečných prací vypsaných na partnerských fakultách. Systém zároveň poskytuje rozhraní pro studenty, kteří se mohou přihlašovat k~tématům, komentovat jednotlivá témata a~závěrečné práce. Dále k~systému přistupuje vedoucí, který schvaluje přihlášení k~tématům a~spravuje stav závěrečných prací. Systém je veřejný a~návštěvníci tak mají možnost procházet jednotlivá témata a~závěrečné práce, dále se mohou registrovat na základě fakultního e-mailu. Hlavní strana systému slouží jako přehled nejnovějších aktivit v~systému.

V~první fázi bylo potřeba vytvořit tzv. \uv{drátěný model webu} (anglicky \textit{wireframe}), který reprezentuje rozvržení jednotlivých komponent na stránce. Jedná se například o~rozdělení hlavní strany na postranní panel, obsahovou část, patičku a~také rozložení tlačítek, formulářů a~tabulek na stránce. Drátěný model vychází z~požadavků zadavatele na systém a~prezentuje konkrétní funkce a~nástroje systému.

Druhá část se zabývá grafickým návrhem, který reprezentuje grafickou podobu uživatelského rozhraní -- jedná se o~vizuální stránku webové aplikace, která navazuje na vytvořený drátěný model. Vytvoření grafického návrhu zahrnuje použití grafických a~typografických prvků. Základním a~velmi důležitým prvkem každé webové aplikace je dobrá typografie. Typografie představuje zejména organizaci písma v~ploše, nepředstavuje tedy například kombinaci barev, nebo dokonce samotný výběr fontu, ale spíš šířku odstavce, velikost a~řez písma, rozdělení nadpisů na podnadpisy atd.

Další část zahrnuje nakódování grafického návrhu tj. prezentaci informací v~definované grafické podobě webovými prohlížeči. Úkolem webdesignu je vytvořit návrh, který zobrazí většina majoritních prohlížečů. Pro usnadnění práce jsem použil webový rámec Twitter Bootstrap postavený na nejmodernějších technologiích HTML5, CSS3 a~JavaScriptu -- tyto technologie slouží pro prezentaci informací a jejich formátování. Twitter Bootstrap je především sada CSS stylů a~javascriptových knihoven, které poskytují rozhraní pro vytvoření základní kostry prezentační vrstvy systému. Rámec dále poskytuje tzv. responzivní design, což je technologie, díky které je možné přizpůsobit rozložení webové stránky dle šířky displeje zařízení.

Informační systém je implementován za použití platformy Grails, která je založena na MVC architektuře (Model--View--Controller). Tato architektura byla navržena Trygvem Reenskaugem na konci sedmdesatých let dvacátého století. Podstatou je explicitní rozdělení na interaktivní prezentaci informací uživateli (View), zpracování uživatelského vstupu (Controller) a~reprezentaci informací, s~nimiž aplikace pracuje (Model). Moje bakalářská práce zahrnuje právě prezentaci informací, k~čemuž využívá platforma Grails technologii GSP (Groovy Server Pages).

Vytvoření informačního systému je jednou z~nejsložitějších programátorských a~návrhářských disciplín. Ve většině případů pokračuje vývoj na systému i~po jeho uvedení do provozu. Některé systémy vyžadují velmi podrobnou analýzu, která není vždy jednoduše proveditelná především z~důvodu špatné komunikace uživatelů systému s~analytiky. V našem případě jde o~relativně jednoduchý systém, jehož použitelnost nebylo příliš složité otestovat. Ani dobře otestovaný systém však není základem úspěchu. Uživatelské rozhraní jsem navrhl s ohledem na možnost jeho budoucího rozšíření, které bude následovat, pokud se prokáže použitelnost systému.
